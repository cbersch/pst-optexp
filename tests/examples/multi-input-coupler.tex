%
% Example from https://tex.stackexchange.com/questions/581952/wdmcoupler-with-multiple-inputs-in-pst-optexp
%
\documentclass[margin=5pt]{standalone}
\usepackage[dvipsnames,svgnames,pdf]{pstricks}
\usepackage{auto-pst-pdf}
\usepackage{pst-optexp}
\usepackage{stackengine}
\begin{document}
\begin{pspicture}(10,6)
  \begin{optexp}
    \psset[optexp]{usefiberstyle, couplersize=0.25, couplersep=0.07}
    \newpsstyle{Fiber}{linecolor=orange, linewidth=2\pslinewidth}
    \pnodes(2, 1) {PumpDiode}
    \pnodes(2, 5) {UpperPumpDiode}
    \pnodes(1, 3) {SignalIsolatorIn} (4, 3) {SignalIsolatorOut}
    \pnodes(1, 3) {SignalIn} (10, 3) {SignalCombinerOut}
    \pnodes(7, 3) {FiberIn} (9, 3) {FiberOut}
    \pnode(9, 3){AmpOut}
    \optdiode[compname=PumpDiode, position=start](PumpDiode)([Xnodesep=1]PumpDiode){Pump diode}
    \optdiode[compname=UpperPumpDiode, position=start](UpperPumpDiode)([Xnodesep=1]UpperPumpDiode){Pump diode}
    \optisolator[compname=SignalIsolator, fiber=none](SignalIsolatorIn)(SignalIsolatorOut)%
        {\begin{tabular}{@{}c@{}}Signal\\Isolator\end{tabular}}
    \wdmcoupler[compname=SignalPumpCombiner](UpperPumpDiode)(\oenodeOut{SignalIsolator})(PumpDiode)(FiberIn)%
        {\begin{tabular}{@{}c@{}}Signal/Pump\\light combiner\end{tabular}}
    \optfiber[
        compname=ActiveFiber,
        position=start,
        addtoFiberOut={arrows=->}](FiberIn)(FiberOut)%
        {\begin{tabular}{@{}c@{}}Active\\fiber\end{tabular}}
    \nput{75}{AmpOut}{\begin{tabular}{@{}c@{}}Amplifier\\Output\end{tabular}}
    \nput{-90}{SignalIn}{\begin{tabular}{@{}c@{}}Signal\\Input\end{tabular}}         
    \drawfiber[ArrowInside=->](SignalIn){SignalIsolator}
  \end{optexp}
\end{pspicture}
\end{document}